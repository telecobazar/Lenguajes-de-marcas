\documentclass[10pt,a4paper]{article}
\usepackage[utf8]{inputenc}
\usepackage{amsmath}
\usepackage{amsfonts}
\usepackage{amssymb}
\usepackage{color}
\begin{document}
	\section{Colores con nombre}
	probamos los siguientes colores:
	\begin{itemize}
	\item Este texto se escribe \textcolor{red} {en rojo}
	\item Este texto se escribe \textcolor{green} {en verde}
	\item Este texto se escribe \textcolor{blue} {en azul}
	\item Este texto se escribe \textcolor{yellow} {en amarillo}
	\item Este texto se escribe \textcolor{magenta} {en magenta}
	\item Este texto se escribe \textcolor{cyan} {en cyan}

	\end{itemize}
	
	
	\section{Colores sin nombre, con RGB}
	probamos los siguientes colores:
	\begin{itemize}
	\item Este texto se escribe \textcolor[rgb] {1,0,0} {en rojo}
	\item Este texto se escribe \textcolor[rgb] {0,1,0} {en verde}
	\item Este texto se escribe \textcolor[rgb] {0,0,1} {en azul}
	\item Este texto se escribe \textcolor[rgb] {1,1,0} {en amarillo}
	\item Este texto se escribe \textcolor[rgb] {1,0,0.6} {en magenta}
	\item Este texto se escribe \textcolor[rgb] {0,0.7,1} {en cyan} 

	\end{itemize}
	
	
	
	\section{Colores sin nombre, con CMYK}
	probamos los siguientes colores:
	\begin{itemize}
	\item Este texto se escribe \textcolor[cmyk] {0,1,1,0} {en rojo} 
	\item Este texto se escribe \textcolor[cmyk] {0.7,0,1,0} {en verde}
	\item Este texto se escribe \textcolor[cmyk] {1,1,0,0} {en azul}
	\item Este texto se escribe \textcolor[cmyk] {0,0,1,0} {en amarillo}
	\item Este texto se escribe \textcolor[cmyk] {0,1,0,0} {en magenta}
	\item Este texto se escribe \textcolor[cmyk] {1,0,0,0} {en cyan}
	

	\end{itemize}
	
\end{document}